% print.tex - Print/PDF version for MAG P≠NP blueprint
% For use with pdflatex or lualatex
% Build: pdflatex print.tex && bibtex print && pdflatex print.tex && pdflatex print.tex

\documentclass[11pt,a4paper]{report}

% ============================================================
% Packages
% ============================================================
\usepackage[utf8]{inputenc}
\usepackage[T1]{fontenc}
\usepackage{amsmath,amssymb,amsthm}
\usepackage{mathtools}
\usepackage{hyperref}
\usepackage{xcolor}
\usepackage{booktabs}
\usepackage{enumitem}
\usepackage{geometry}
\usepackage{longtable}
\usepackage{url}

\geometry{margin=1in}

% ============================================================
% Hyperref configuration
% ============================================================
\hypersetup{
  colorlinks=true,
  linkcolor=blue!60!black,
  citecolor=green!50!black,
  urlcolor=blue!70!black,
  pdftitle={MAG P≠NP: A Group-Theoretic Approach - Lean 4 Formalization Blueprint},
  pdfauthor={Masaru Numagaki}
}

% ============================================================
% Load common mathematical macros
% ============================================================
% macros/common.tex - Common mathematical macros for MAG P≠NP blueprint
% Shared between web and print versions

% ============================================================
% Mathematical Notation
% ============================================================

% Groups
\newcommand{\Sym}[1]{\mathrm{Sym}(#1)}
\newcommand{\Alt}[1]{\mathrm{Alt}(#1)}
\newcommand{\Perm}[1]{\mathrm{Perm}(#1)}
\newcommand{\Afive}{A_5}
\newcommand{\Sfive}{S_5}

% Complexity classes
\newcommand{\PMAG}{\mathrm{P}_{\mathrm{MAG}}}
\newcommand{\NPMAG}{\mathrm{NP}_{\mathrm{MAG}}}
\newcommand{\NCone}{\mathrm{NC}^1}

% Predicates
\newcommand{\IsSolvable}{\mathrm{IsSolvable}}
\newcommand{\IsMAGPolyTime}{\mathrm{IsMAGPolyTime}}
\newcommand{\IsMAGNPHard}{\mathrm{IsMAGNPHard}}
\newcommand{\Solves}{\mathrm{Solves}}

% Structures
\newcommand{\derivedSeries}{\mathrm{derivedSeries}}
\newcommand{\CommTerm}{\mathrm{CommTerm}}

% ============================================================
% Common Abbreviations
% ============================================================

% Group isomorphism
\newcommand{\iso}{\simeq}

% Injective embedding
\newcommand{\injects}{\hookrightarrow}

% Commutator
\newcommand{\comm}[2]{[#1, #2]}

% ============================================================
% Lean-related
% ============================================================

% Mathlib reference
\newcommand{\mathlib}{\textsc{Mathlib}}

% sorry-free indicator
\newcommand{\sorryfree}{\textcolor{green!60!black}{sorry-free}}


% ============================================================
% Blueprint macros for print version
% (These ARE defined here because leanblueprint plugin
%  only processes web output, not PDF)
% ============================================================

% Lean code reference - prints in monospace for print version
\newcommand{\lean}[1]{\texttt{#1}}

% Mark as formalized in Lean
\newcommand{\leanok}{\hfill\textcolor{green!60!black}{$\checkmark$~Lean}}

% Mark as not yet formalized
\newcommand{\notready}{\hfill\textcolor{red!60!black}{$\times$~TODO}}

% Dependency declaration (no-op in print, info is in web graph)
\newcommand{\uses}[1]{}
\newcommand{\proves}[1]{}

% Paper reference (section correspondence)
\newcommand{\paperref}[1]{\textit{(Paper: #1)}}

% Project URLs (no-op in print)
\newcommand{\home}[1]{}
\newcommand{\github}[1]{}
\newcommand{\dochome}[1]{}

% ============================================================
% Theorem environments
% ============================================================
\theoremstyle{definition}
\newtheorem{definition}{Definition}[section]
\newtheorem{example}[definition]{Example}

\theoremstyle{plain}
\newtheorem{theorem}[definition]{Theorem}
\newtheorem{lemma}[definition]{Lemma}
\newtheorem{corollary}[definition]{Corollary}
\newtheorem{proposition}[definition]{Proposition}

\theoremstyle{remark}
\newtheorem{remark}[definition]{Remark}

% ============================================================
% Document info
% ============================================================
\title{\textbf{MAG P$\neq$NP: A Group-Theoretic Approach}\\[1em]
       \large Lean 4 Formalization Blueprint\\[0.5em]
       \normalsize Version 1.0}
\author{Masaru Numagaki}
\date{January 2026}

% ============================================================
% Document
% ============================================================
\begin{document}

\maketitle

\begin{abstract}
This blueprint documents the Lean 4 formalization of P$\neq$NP in the MAG (Minimum Asymmetry Gap) model.
The MAG framework translates computational complexity into group theory:
\textbf{polytime} corresponds to solvable groups,
\textbf{NP-hard} corresponds to non-solvable groups (with $A_5$ as the minimal non-solvable core), and
\textbf{solves} corresponds to group isomorphism.

\medskip
\noindent\textbf{Main Results (all sorry-free):}
\begin{enumerate}
  \item \textbf{Theorem 3.1}: MAG internal separation (unconditional)
  \item \textbf{Theorem 4.6}: $A_5$ Universal Barrier (unconditional)
  \item \textbf{Theorem 4.7}: Solvable vanishing depth (unconditional)
  \item \textbf{Theorem 5.4}: Bridge theorem (conditional on TranslationInterface)
  \item \textbf{Theorem 5.7}: Toy model end-to-end verification (unconditional)
\end{enumerate}

\medskip
\noindent\textbf{Proof Status:} Core/ directory: sorry = 0, axiom = 0
\end{abstract}

\tableofcontents

% Include the main content
% content.tex for MAG P≠NP project (leanblueprint format)
% Author: Masaru Numagaki
% Date: January 2026
% Version: 1.3 (aligned with mag_pnp.pdf references [1]-[12], added Klein)
%
% Usage: This file is designed for Patrick Massot's leanblueprint tool.
% See https://github.com/PatrickMassot/leanblueprint
%
% Correspondence: This blueprint follows the structure of
% "MAG: Formalizing the Solvability Barrier in P vs NP via Group-Theoretic Semantics"
%
% Bibliography keys correspond to references.bib:
%   [1] Barrington1986    [5] RhodesSteinberg2009  [9] AroraBarak2009
%   [2] Cook1971          [6] Lean4_2021           [10] FeitThompson1963
%   [3] Galois1846        [7] Mathlib2020          [11] CFSG1994
%   [4] Klein1872         [8] Rotman2012           [12] KrohnRhodes1965

%%%%%%%%%%%%%%%%%%%%%%%%%%%%%%%%%%%%%%%%%%%%%%%%%%%%%%%%%%%%%%%%%%%%%%%%%%%%%
\chapter{Introduction}
\label{chap:introduction}

This blueprint presents a formal verification of P$\neq$NP in the MAG (Minimum Asymmetry Gap) model,
implemented in Lean~4 \cite{Lean4_2021} with Mathlib \cite{Mathlib2020}. The MAG framework translates computational complexity into group theory:
\begin{itemize}
  \item \textbf{Polytime} corresponds to \emph{solvable groups}
  \item \textbf{NP-hard} corresponds to \emph{non-solvable groups} (with $A_5$ as the minimal non-solvable core)
  \item \textbf{Solves} corresponds to \emph{group isomorphism}
\end{itemize}

\section{Main Results}

The main formalized results are:
\begin{enumerate}
  \item \textbf{MAG internal separation} (Theorem 3.1): Unconditional, fully formalized
  \item \textbf{$A_5$ Universal Barrier} (Theorem 4.6): Unconditional, fully formalized
  \item \textbf{Solvable vanishing depth} (Theorem 4.7): Unconditional, fully formalized
  \item \textbf{Bridge theorem} (Theorem 5.4): Conditional on TranslationInterface, fully formalized
  \item \textbf{Toy model verification} (Theorem 5.7): End-to-end, unconditional, fully formalized
\end{enumerate}

\section{Scope and Non-claims}

\begin{remark}[Scope]
This formalization does NOT unconditionally claim to solve the Clay P vs NP problem.
What we prove:
\begin{itemize}
  \item \textbf{Unconditional:} MAG-internal P$\neq$NP, where definitions are group-theoretic
  \item \textbf{Conditional:} If TranslationInterface holds for model $M$, then P$\neq$NP in $M$
\end{itemize}
The TranslationInterface separates the formalized core from unformalized translation assumptions.
\end{remark}

\section{Historical Foundation}

The MAG framework can be understood as the \textbf{computational analogue of Klein's Erlangen Program}.
Just as Klein (1872) proposed classifying geometries by their transformation groups,
MAG classifies computational problems by their symmetry groups and algorithms by their operation groups.

The MAG framework builds on several foundational results:
\begin{itemize}
  \item \textbf{Galois Theory} \cite{Galois1846}: The non-solvability of $A_5$ and the connection between solvability and radical solutions
  \item \textbf{Klein's Erlangen Program} \cite{Klein1872}: The methodology of translating mathematical objects into group-theoretic objects for classification
  \item \textbf{Krohn-Rhodes Decomposition} \cite{KrohnRhodes1965, RhodesSteinberg2009}: Every finite automaton decomposes into simple groups and aperiodic monoids
  \item \textbf{Barrington's Theorem} \cite{Barrington1986}: NC$^1$ = width-5 branching programs over $S_5$
  \item \textbf{Cook-Levin Theorem} \cite{Cook1971}: SAT is NP-complete, establishing the existence of NP-hard problems
  \item \textbf{Classification of Finite Simple Groups} \cite{CFSG1994}: $A_5$ is the smallest non-solvable simple group
  \item \textbf{Feit-Thompson Theorem} \cite{FeitThompson1963}: All groups of odd order are solvable
\end{itemize}

For a comprehensive treatment of computational complexity, see Arora and Barak \cite{AroraBarak2009}.
For group theory foundations, see Rotman \cite{Rotman2012}.

\section{File Structure}

\begin{center}
\begin{tabular}{|l|l|l|}
\hline
\textbf{Paper Section} & \textbf{Lean File} & \textbf{Status} \\
\hline
Section 2 (Definitions) & \texttt{Core/MAG\_P\_neq\_NP.lean} & $\checkmark$ sorry-free \\
Section 3 (Main Theorem) & \texttt{Core/MAG\_P\_neq\_NP.lean} & $\checkmark$ sorry-free \\
Section 4.1--4.5 & \texttt{Core/A5\_Barrier.lean} & $\checkmark$ sorry-free \\
Section 4.6 & \texttt{Core/A5\_Barrier.lean} & $\checkmark$ sorry-free \\
Section 4.7 & \texttt{Core/BarringtonBarrier.lean} & $\checkmark$ sorry-free \\
Section 5.1--5.3 & \texttt{Core/Bridge/Interface.lean} & $\checkmark$ sorry-free \\
Section 5.4 & \texttt{Core/Bridge/Theorem.lean} & $\checkmark$ sorry-free \\
Section 5.5 & \texttt{Core/Bridge/ToyInstantiation.lean} & $\checkmark$ sorry-free \\
Appendix A & \texttt{Support/Barrington/*.lean} & $\checkmark$ sorry-free \\
Appendix A & \texttt{Support/Circuits/KrohnRhodes.lean} & $\checkmark$ sorry-free \\
Appendix A & \texttt{Support/Syntactic/Theory.lean} & $\checkmark$ sorry-free \\
Appendix B & \texttt{Blueprint/*} & Design docs \\
\hline
\end{tabular}
\end{center}

%%%%%%%%%%%%%%%%%%%%%%%%%%%%%%%%%%%%%%%%%%%%%%%%%%%%%%%%%%%%%%%%%%%%%%%%%%%%%
\chapter{MAG Model Definitions}
\label{chap:mag-model}

This chapter introduces the core definitions of the MAG framework.
\paperref{Section 2}

%%%%%%%%%%%%%%%%%%%%%%%%%%%%%%%%%%%%%%%%%%%%%%%%%%%%%%%%%%%%%%%%%%%%%%%%%%%%%
\section{Problem and Algorithm Structures}

\begin{definition}[ProblemInstance]
\label{def:problem-instance}
\lean{MAG.Core.ProblemInstance}
\leanok
\paperref{Definition 2.1}

A \emph{problem instance} is represented by its symmetry group $G$.
\begin{verbatim}
structure ProblemInstance where
  G : Type
  [group_G : Group G]
  [fintype_G : Fintype G]
\end{verbatim}
\end{definition}

\begin{definition}[Algorithm]
\label{def:algorithm}
\lean{MAG.Core.Algorithm}
\leanok
\paperref{Definition 2.2}

An \emph{algorithm} is represented by the group of operations it can generate.
\begin{verbatim}
structure Algorithm where
  G_alg : Type
  [group_alg : Group G_alg]
  [fintype_alg : Fintype G_alg]
\end{verbatim}
\end{definition}

%%%%%%%%%%%%%%%%%%%%%%%%%%%%%%%%%%%%%%%%%%%%%%%%%%%%%%%%%%%%%%%%%%%%%%%%%%%%%
\section{MAG Complexity Classes}

\begin{definition}[IsMAGPolyTime]
\label{def:mag-polytime}
\lean{MAG.Core.IsMAGPolyTime}
\leanok
\uses{def:algorithm}
\paperref{Definition 2.3}

An algorithm is \emph{MAG-polytime} if its operation group is solvable.
This definition is justified by the Krohn-Rhodes decomposition theorem \cite{KrohnRhodes1965}:
standard Boolean gates decompose into solvable components.
\[
  a \in \mathrm{P}_{\mathrm{MAG}} \quad:\Longleftrightarrow\quad A(a)\ \text{is solvable}
\]
\begin{verbatim}
class IsMAGPolyTime (alg : Algorithm) : Prop where
  solvable_structure : IsSolvable alg.G_alg
\end{verbatim}
\end{definition}

\begin{definition}[IsMAGNPHard]
\label{def:mag-nphard}
\lean{MAG.Core.IsMAGNPHard}
\leanok
\uses{def:problem-instance}
\paperref{Definition 2.4}

A problem is \emph{MAG-NP-hard} if its symmetry group is non-solvable.
\[
  p\ \text{is MAG-NP-hard} \quad:\Longleftrightarrow\quad \neg\, \mathrm{IsSolvable}(G(p))
\]
\begin{verbatim}
class IsMAGNPHard (prob : ProblemInstance) : Prop where
  non_solvable : ¬ IsSolvable prob.G
\end{verbatim}
\end{definition}

\begin{definition}[Solves]
\label{def:solves}
\lean{MAG.Core.Solves}
\leanok
\uses{def:algorithm, def:problem-instance}
\paperref{Definition 2.5}

An algorithm \emph{solves} a problem if their groups are isomorphic.
\[
  \mathrm{Solves}(a,p) \quad:\Longleftrightarrow\quad A(a) \simeq G(p) \quad\text{(group isomorphism)}
\]
\begin{verbatim}
def Solves (alg : Algorithm) (prob : ProblemInstance) : Prop :=
  Nonempty (alg.G_alg ≃* prob.G)
\end{verbatim}
\end{definition}

\begin{definition}[MAG\_P\_equals\_NP]
\label{def:mag-p-equals-np}
\lean{MAG.Core.MAG\_P\_equals\_NP}
\leanok
\uses{def:mag-polytime, def:mag-nphard, def:solves}
\paperref{Definition 2.6}

The MAG version of P=NP states that every MAG-NP-hard problem can be solved by some MAG-polytime algorithm.
\begin{verbatim}
def MAG_P_equals_NP : Prop :=
  ∀ (prob : ProblemInstance) [IsMAGNPHard prob],
    ∃ (alg : Algorithm), IsMAGPolyTime alg ∧ Solves alg prob
\end{verbatim}
\end{definition}

%%%%%%%%%%%%%%%%%%%%%%%%%%%%%%%%%%%%%%%%%%%%%%%%%%%%%%%%%%%%%%%%%%%%%%%%%%%%%
\section{The $A_5$ Instance}

\begin{definition}[A5\_Instance]
\label{def:a5-instance}
\lean{MAG.Core.A5\_Instance}
\leanok
\uses{def:problem-instance}

The alternating group $A_5$ as a problem instance.
\begin{verbatim}
def A5_Instance : ProblemInstance where
  G := alternatingGroup (Fin 5)
\end{verbatim}
\end{definition}

%%%%%%%%%%%%%%%%%%%%%%%%%%%%%%%%%%%%%%%%%%%%%%%%%%%%%%%%%%%%%%%%%%%%%%%%%%%%%
\chapter{Core Algebraic Facts}
\label{chap:algebraic-facts}

%%%%%%%%%%%%%%%%%%%%%%%%%%%%%%%%%%%%%%%%%%%%%%%%%%%%%%%%%%%%%%%%%%%%%%%%%%%%%
\section{Properties of $A_5$}

\begin{theorem}[$A_5$ is non-solvable]
\label{thm:a5-not-solvable}
\lean{MAG.Core.A5\_not\_solvable}
\leanok

The alternating group $A_5$ is not solvable:
\[
  \neg\, \mathrm{IsSolvable}(A_5)
\]
This is the fundamental algebraic fact underlying MAG separation, first established by Galois \cite{Galois1846}.
The proof uses the simplicity of $A_5$ and explicit non-commuting witnesses.
\end{theorem}

\begin{theorem}[$A_5$ is simple]
\label{thm:a5-simple}
\lean{MAG.Core.A5\_is\_simple}
\leanok

$A_5$ is a simple group (from Mathlib: \texttt{alternatingGroup.isSimpleGroup\_five}).
\end{theorem}

\begin{theorem}[$|A_5| = 60$]
\label{thm:a5-card}
\lean{MAG.Core.A5\_card}
\leanok

The cardinality of $A_5$ is exactly 60 (proven via \texttt{native\_decide}).
By the classification of finite simple groups \cite{CFSG1994}, $A_5$ is the smallest non-solvable simple group.
The Feit-Thompson theorem \cite{FeitThompson1963} further establishes that non-solvability requires even order.
\end{theorem}

\begin{corollary}[$A_5$ is MAG-NP-hard]
\label{cor:a5-nphard}
\lean{MAG.Core.A5\_is\_NPHard}
\leanok
\uses{def:a5-instance, def:mag-nphard, thm:a5-not-solvable}

Since $A_5$ is non-solvable, the $A_5$ instance is MAG-NP-hard.
\end{corollary}

%%%%%%%%%%%%%%%%%%%%%%%%%%%%%%%%%%%%%%%%%%%%%%%%%%%%%%%%%%%%%%%%%%%%%%%%%%%%%
\section{Solvability Preservation}

\begin{lemma}[Isomorphism preserves solvability]
\label{lem:solvable-of-equiv}
\lean{MAG.Core.solvable\_of\_equiv}
\leanok
\paperref{Lemma 2.7}

If $G \simeq H$ (group isomorphism) and $G$ is solvable, then $H$ is solvable.
\[
  G \simeq H \land \mathrm{IsSolvable}(G) \implies \mathrm{IsSolvable}(H)
\]
\end{lemma}

%%%%%%%%%%%%%%%%%%%%%%%%%%%%%%%%%%%%%%%%%%%%%%%%%%%%%%%%%%%%%%%%%%%%%%%%%%%%%
\chapter{MAG Separation Theorem}
\label{chap:mag-separation}

\paperref{Section 3}

%%%%%%%%%%%%%%%%%%%%%%%%%%%%%%%%%%%%%%%%%%%%%%%%%%%%%%%%%%%%%%%%%%%%%%%%%%%%%
\section{Main Theorem}

\begin{theorem}[MAG P$\neq$NP --- Theorem 3.1]
\label{thm:mag-p-neq-np}
\lean{MAG.Core.MAG\_P\_neq\_NP}
\leanok
\uses{def:mag-p-equals-np, cor:a5-nphard, lem:solvable-of-equiv}
\paperref{Theorem 3.1}

P $\neq$ NP in the MAG model:
\[
  \neg\, \mathrm{MAG\_P\_equals\_NP}
\]

\textbf{Proof sketch:}
\begin{enumerate}
  \item Assume MAG P = NP for contradiction.
  \item Take $A_5$ as an NP-hard instance (non-solvable).
  \item By assumption, there exists a polytime algorithm $a$ solving $A_5$.
  \item $a$ is polytime $\Rightarrow$ $A(a)$ is solvable.
  \item $a$ solves $A_5$ $\Rightarrow$ $A(a) \simeq A_5$.
  \item By Lemma~\ref{lem:solvable-of-equiv}, $A_5$ would be solvable.
  \item Contradiction with Theorem~\ref{thm:a5-not-solvable}.
\end{enumerate}

\textbf{Status:} sorry = 0, axiom = 0
\end{theorem}

%%%%%%%%%%%%%%%%%%%%%%%%%%%%%%%%%%%%%%%%%%%%%%%%%%%%%%%%%%%%%%%%%%%%%%%%%%%%%
\chapter{$A_5$ Universal Barrier}
\label{chap:a5-barrier}

\paperref{Section 4}

%%%%%%%%%%%%%%%%%%%%%%%%%%%%%%%%%%%%%%%%%%%%%%%%%%%%%%%%%%%%%%%%%%%%%%%%%%%%%
\section{Syntactic Systems and Deep Structures}

\begin{definition}[SyntacticSystem]
\label{def:syntactic-system}
\lean{MAG.Core.SyntacticSystem}
\leanok
\paperref{Definition 4.1}

A \emph{syntactic system} represents a formal derivation system with solvable group structure.
\begin{verbatim}
structure SyntacticSystem where
  Derivations : Type*
  [group : Group Derivations]
  [fintype : Fintype Derivations]
  is_solvable : IsSolvable Derivations
\end{verbatim}
\end{definition}

\begin{definition}[DeepStructure]
\label{def:deep-structure}
\lean{MAG.Core.DeepStructure}
\leanok
\paperref{Definition 4.2}

A \emph{deep structure} contains an $A_5$ embedding via injective homomorphism.
\begin{verbatim}
structure DeepStructure where
  SymmetryGroup : Type*
  [group : Group SymmetryGroup]
  [fintype : Fintype SymmetryGroup]
  embedding : alternatingGroup (Fin 5) →* SymmetryGroup
  embedding_injective : Function.Injective embedding
\end{verbatim}
\end{definition}

\begin{definition}[CanDescribe]
\label{def:can-describe}
\lean{MAG.Core.CanDescribe}
\leanok
\uses{def:syntactic-system, def:deep-structure}
\paperref{Definition 4.3}

A syntactic system \emph{can describe} a deep structure if there exists a group isomorphism.
\begin{verbatim}
def CanDescribe (sys : SyntacticSystem) (deep : DeepStructure) : Prop :=
  Nonempty (sys.Derivations ≃* deep.SymmetryGroup)
\end{verbatim}
\end{definition}

%%%%%%%%%%%%%%%%%%%%%%%%%%%%%%%%%%%%%%%%%%%%%%%%%%%%%%%%%%%%%%%%%%%%%%%%%%%%%
\section{Kernel Propagation}

\begin{lemma}[Non-solvable kernel propagates]
\label{lem:nonsolvable-propagates}
\lean{MAG.Core.nonsolvable\_of\_injective\_from\_nonsolvable}
\leanok
\paperref{Lemma 4.4}

If $H$ is non-solvable and $f : H \hookrightarrow G$ is injective, then $G$ is non-solvable.
\[
  \neg\mathrm{IsSolvable}(H) \land (H \hookrightarrow G) \implies \neg\mathrm{IsSolvable}(G)
\]
\end{lemma}

\begin{corollary}[$A_5$ embedding implies non-solvability]
\label{cor:a5-embedding-nonsolvable}
\lean{MAG.Core.A5\_embedding\_implies\_nonsolvable}
\leanok
\uses{def:deep-structure, lem:nonsolvable-propagates, thm:a5-not-solvable}
\paperref{Lemma 4.5}

Any deep structure (with $A_5$ embedding) has a non-solvable symmetry group.
\end{corollary}

%%%%%%%%%%%%%%%%%%%%%%%%%%%%%%%%%%%%%%%%%%%%%%%%%%%%%%%%%%%%%%%%%%%%%%%%%%%%%
\section{Universal Barrier Theorem}

\begin{theorem}[$A_5$ Universal Barrier --- Theorem 4.6]
\label{thm:a5-universal-barrier}
\lean{MAG.Core.A5\_Universal\_Barrier}
\leanok
\uses{def:syntactic-system, def:deep-structure, def:can-describe, cor:a5-embedding-nonsolvable}
\paperref{Theorem 4.6}

No solvable syntactic system can describe a deep structure containing an $A_5$ embedding:
\[
  \forall S\ D,\ \neg\, \mathrm{CanDescribe}(S, D)
\]

This theorem reflects the Krohn-Rhodes complexity hierarchy \cite{KrohnRhodes1965, RhodesSteinberg2009}:
$A_5$ is the minimal Level~2 group, unreachable from Level~0--1 structures.

\textbf{Status:} sorry = 0, axiom = 0
\end{theorem}

%%%%%%%%%%%%%%%%%%%%%%%%%%%%%%%%%%%%%%%%%%%%%%%%%%%%%%%%%%%%%%%%%%%%%%%%%%%%%
\chapter{Commutator Tree Vanishing Depth}
\label{chap:barrington-barrier}

\paperref{Section 4.7}

%%%%%%%%%%%%%%%%%%%%%%%%%%%%%%%%%%%%%%%%%%%%%%%%%%%%%%%%%%%%%%%%%%%%%%%%%%%%%
\section{Commutator Trees}

\begin{definition}[CommTerm]
\label{def:comm-term}
\lean{MAG.Core.BarringtonBarrier.CommTerm}
\leanok

A balanced commutator tree of depth $n$:
\begin{itemize}
  \item Depth 0: a variable
  \item Depth $n+1$: commutator $[t, u]$ of two depth-$n$ terms
\end{itemize}
\begin{verbatim}
inductive CommTerm (α : Type u) : ℕ → Type u
  | var : α → CommTerm α 0
  | comm {n : ℕ} : CommTerm α n → CommTerm α n → CommTerm α (n+1)
\end{verbatim}
\end{definition}

\begin{theorem}[Evaluation lands in derived series]
\label{thm:eval-mem-derived}
\lean{MAG.Core.BarringtonBarrier.CommTerm.eval\_mem\_derivedSeries}
\leanok
\uses{def:comm-term}

For any commutator term $t$ of depth $n$, $\mathrm{eval}(t) \in \mathrm{derivedSeries}(G, n)$.
\end{theorem}

%%%%%%%%%%%%%%%%%%%%%%%%%%%%%%%%%%%%%%%%%%%%%%%%%%%%%%%%%%%%%%%%%%%%%%%%%%%%%
\section{Vanishing Depth}

\begin{theorem}[Solvable vanishing depth --- Theorem 4.7]
\label{thm:solvable-vanishing-depth}
\lean{MAG.Core.BarringtonBarrier.CommTerm.solvable\_vanishing\_depth}
\leanok
\uses{def:comm-term, thm:eval-mem-derived}
\paperref{Theorem 4.7}

For any solvable group $G$, there exists $k$ such that all commutator trees of depth $n \geq k$ evaluate to 1:
\[
  \exists k,\ \forall n \geq k,\ \forall t : \mathrm{CommTerm}_n,\ \mathrm{eval}(t) = 1
\]

\textbf{Status:} sorry = 0, axiom = 0
\end{theorem}

\begin{theorem}[$A_5$ has no vanishing depth]
\label{thm:a5-no-vanishing}
\lean{MAG.Core.BarringtonBarrier.A5\_no\_vanishing\_depth}
\leanok

$A_5$ has no finite vanishing depth (its derived series is constantly $\top$).
\end{theorem}

%%%%%%%%%%%%%%%%%%%%%%%%%%%%%%%%%%%%%%%%%%%%%%%%%%%%%%%%%%%%%%%%%%%%%%%%%%%%%
\chapter{Translation Interface and Bridge}
\label{chap:bridge}

\paperref{Section 5}

%%%%%%%%%%%%%%%%%%%%%%%%%%%%%%%%%%%%%%%%%%%%%%%%%%%%%%%%%%%%%%%%%%%%%%%%%%%%%
\section{Abstract Computational Model}

\begin{definition}[ComputationalModel]
\label{def:computational-model}
\lean{MAG.Core.Bridge.ComputationalModel}
\leanok
\paperref{Definition 5.1}

An abstract interface for any computational model:
\begin{verbatim}
class ComputationalModel (M : Type*) where
  Problem : Type*
  Algorithm : Type*
  problemSymmetry : Problem → Type*
  algorithmOperations : Algorithm → Type*
  solves : Algorithm → Problem → Prop
  isPolyTime : Algorithm → Prop
  isNPHard : Problem → Prop
\end{verbatim}
\end{definition}

%%%%%%%%%%%%%%%%%%%%%%%%%%%%%%%%%%%%%%%%%%%%%%%%%%%%%%%%%%%%%%%%%%%%%%%%%%%%%
\section{Translation Interface}

\begin{definition}[TranslationInterface]
\label{def:translation-interface}
\lean{MAG.Core.Bridge.TranslationInterface}
\leanok
\uses{def:computational-model}
\paperref{Definition 5.2}

The interface connecting a computational model to MAG semantics:
\begin{enumerate}
  \item \textbf{(A) polyTimeIsSolvable:} Polytime algorithms have solvable operation groups
        (justified by Krohn-Rhodes \cite{KrohnRhodes1965})
  \item \textbf{(B) npHardHasA5:} NP-hard problems have $A_5$ embedding in symmetry
        (justified by Barrington \cite{Barrington1986} + syntactic theory)
  \item \textbf{(C) solvesImpliesGroupCorrespondence:} Solving implies group isomorphism
\end{enumerate}
\end{definition}

%%%%%%%%%%%%%%%%%%%%%%%%%%%%%%%%%%%%%%%%%%%%%%%%%%%%%%%%%%%%%%%%%%%%%%%%%%%%%
\section{Standard P=NP Definition}

\begin{definition}[Standard\_P\_equals\_NP]
\label{def:standard-p-equals-np}
\lean{MAG.Core.Bridge.Standard\_P\_equals\_NP}
\leanok
\uses{def:computational-model}
\paperref{Definition 5.3}

P = NP on model $M$: every NP-hard problem can be solved by some polytime algorithm.
\begin{verbatim}
def Standard_P_equals_NP (M : Type*) [ComputationalModel M] : Prop :=
  ∀ (p : Problem), isNPHard p →
    ∃ a : Algorithm, isPolyTime a ∧ solves a p
\end{verbatim}
\end{definition}

%%%%%%%%%%%%%%%%%%%%%%%%%%%%%%%%%%%%%%%%%%%%%%%%%%%%%%%%%%%%%%%%%%%%%%%%%%%%%
\section{Bridge Theorem}

\begin{theorem}[Bridge P$\neq$NP --- Theorem 5.4]
\label{thm:bridge-p-neq-np}
\lean{MAG.Core.Bridge.Bridge\_P\_neq\_NP}
\leanok
\uses{def:computational-model, def:translation-interface, def:standard-p-equals-np}
\paperref{Theorem 5.4}

If TranslationInterface holds for model $M$ and an NP-hard problem exists, then P $\neq$ NP:
\[
  \mathrm{TranslationInterface}(M) \land (\exists p,\ \mathrm{isNPHard}(p)) \implies \neg\, \mathrm{Standard\_P\_equals\_NP}(M)
\]

\textbf{Status:} sorry = 0, axiom = 0
\end{theorem}

\begin{corollary}[Bridge P$\neq$NP (with existence)]
\label{cor:bridge-p-neq-np-prime}
\lean{MAG.Core.Bridge.Bridge\_P\_neq\_NP'}
\leanok
\uses{thm:bridge-p-neq-np}

Variant that takes existence of NP-hard problem as explicit hypothesis.
\end{corollary}

%%%%%%%%%%%%%%%%%%%%%%%%%%%%%%%%%%%%%%%%%%%%%%%%%%%%%%%%%%%%%%%%%%%%%%%%%%%%%
\chapter{Toy Model Verification}
\label{chap:toy}

\paperref{Section 5.5}

%%%%%%%%%%%%%%%%%%%%%%%%%%%%%%%%%%%%%%%%%%%%%%%%%%%%%%%%%%%%%%%%%%%%%%%%%%%%%
\section{BarringtonToy Model}

\begin{definition}[BarringtonToyModel]
\label{def:barrington-toy-model}
\lean{MAG.Core.Bridge.BarringtonToyModel}
\leanok
\paperref{Construction 5.5}

A toy computational model where:
\begin{itemize}
  \item Problems ARE their symmetry groups
  \item Algorithms ARE their operation groups
  \item Conditions are satisfied BY DEFINITION
\end{itemize}
\end{definition}

\begin{theorem}[Toy satisfies TranslationInterface]
\label{thm:toy-translation-interface}
\lean{MAG.Core.Bridge.instTranslationInterfaceBarringtonToyModel}
\leanok
\uses{def:barrington-toy-model, def:translation-interface}

The BarringtonToy model satisfies all three TranslationInterface conditions.
\end{theorem}

%%%%%%%%%%%%%%%%%%%%%%%%%%%%%%%%%%%%%%%%%%%%%%%%%%%%%%%%%%%%%%%%%%%%%%%%%%%%%
\section{NP-Hard Witness}

\begin{definition}[$S_5$ Problem]
\label{def:s5-problem}
\lean{MAG.Core.Bridge.S5\_Problem}
\leanok
\uses{def:barrington-toy-model}
\paperref{Lemma 5.6}

A toy problem whose symmetry group is $S_5 = \mathrm{Perm}(\mathrm{Fin}\ 5)$.
\end{definition}

\begin{theorem}[$S_5$ is NP-hard in toy model]
\label{thm:s5-nphard}
\lean{MAG.Core.Bridge.S5\_is\_nphard}
\leanok
\uses{def:s5-problem}

The $S_5$ problem is NP-hard in the toy sense (contains $A_5$ via subgroup inclusion).
\end{theorem}

\begin{theorem}[NP-hard problems exist in toy model]
\label{thm:toy-exists-nphard}
\lean{MAG.Core.Bridge.toy\_exists\_nphard}
\leanok
\uses{thm:s5-nphard}

There exists an NP-hard problem in the BarringtonToy model.
\end{theorem}

%%%%%%%%%%%%%%%%%%%%%%%%%%%%%%%%%%%%%%%%%%%%%%%%%%%%%%%%%%%%%%%%%%%%%%%%%%%%%
\section{End-to-End Result}

\begin{theorem}[BarringtonToy P$\neq$NP --- Theorem 5.7]
\label{thm:barrington-toy-p-neq-np}
\lean{MAG.Core.Bridge.BarringtonToy\_P\_neq\_NP}
\leanok
\uses{thm:toy-translation-interface, thm:toy-exists-nphard, cor:bridge-p-neq-np-prime}

P $\neq$ NP holds in the BarringtonToy model:
\[
  \neg\, \mathrm{Standard\_P\_equals\_NP}(\mathrm{BarringtonToyModel})
\]
This theorem demonstrates the bridge plumbing works end-to-end without sorry.

\textbf{Status:} sorry = 0, axiom = 0
\end{theorem}

%%%%%%%%%%%%%%%%%%%%%%%%%%%%%%%%%%%%%%%%%%%%%%%%%%%%%%%%%%%%%%%%%%%%%%%%%%%%%
\chapter{Barrington's Theorem Components}
\label{chap:barrington}

\paperref{Appendix A}

This chapter formalizes the core algebraic content of Barrington's theorem \cite{Barrington1986}.

%%%%%%%%%%%%%%%%%%%%%%%%%%%%%%%%%%%%%%%%%%%%%%%%%%%%%%%%%%%%%%%%%%%%%%%%%%%%%
\section{Non-Commuting Elements}

\begin{definition}[commutator]
\label{def:commutator}
\lean{MAG.Support.Barrington.Complete.commutator}
\leanok

Group commutator: $[a, b] = a^{-1}b^{-1}ab$.
\end{definition}

\begin{theorem}[$\sigma$ and $\tau$ do not commute]
\label{thm:sigma-tau-not-commute}
\lean{MAG.Support.Barrington.Complete.σ\_τ\_not\_commute}
\leanok
\uses{def:commutator}

$[\sigma, \tau] \neq 1$, proven via \texttt{decide}.
\end{theorem}

\begin{theorem}[AND via commutator]
\label{thm:commutator-and-table}
\lean{MAG.Support.Barrington.Complete.commutator\_and\_table}
\leanok
\uses{thm:sigma-tau-not-commute}

The commutator encodes AND truth table:
\[
  [\sigma, \tau] \neq 1,\quad [\sigma, 1] = 1,\quad [1, \tau] = 1,\quad [1, 1] = 1
\]
\end{theorem}

%%%%%%%%%%%%%%%%%%%%%%%%%%%%%%%%%%%%%%%%%%%%%%%%%%%%%%%%%%%%%%%%%%%%%%%%%%%%%
\section{Branching Programs and Length Analysis}

\begin{definition}[BranchingProgram]
\label{def:branching-program}
\lean{MAG.Support.Barrington.Complete.BranchingProgram}
\leanok

A branching program as a list of instructions.
\end{definition}

\begin{theorem}[Commutator program length]
\label{thm:length-commutator-prog}
\lean{MAG.Support.Barrington.Complete.length\_commutatorProg}
\leanok
\uses{def:branching-program}

$|[P_1, P_2]| = 2|P_1| + 2|P_2|$.
\end{theorem}

\begin{theorem}[Translation length bound]
\label{thm:translate-length-bound}
\lean{MAG.Support.Barrington.Complete.translate\_length\_bound}
\leanok

$|\mathrm{translate}(C)| \leq 4^{\mathrm{depth}(C)}$.
\end{theorem}

\begin{theorem}[NC$^1$ has poly BP]
\label{thm:nc1-poly-bp}
\lean{MAG.Support.Barrington.Arithmetic.NC1\_has\_poly\_BP}
\leanok

NC$^1$ functions have polynomial-size width-5 branching programs.
\end{theorem}

%%%%%%%%%%%%%%%%%%%%%%%%%%%%%%%%%%%%%%%%%%%%%%%%%%%%%%%%%%%%%%%%%%%%%%%%%%%%%
\chapter{Krohn-Rhodes Circuit Solvability}
\label{chap:krohn-rhodes}

\paperref{Appendix A}

This chapter proves that standard Boolean circuits have solvable operation groups,
based on the Krohn-Rhodes decomposition theorem \cite{KrohnRhodes1965, RhodesSteinberg2009}.

\begin{definition}[GateType]
\label{def:gate-type}
\lean{MAG.Support.Circuits.KrohnRhodes.GateType}
\leanok

Standard Boolean gate types: AND, OR, NOT, INPUT.
\end{definition}

\begin{definition}[GateGroup]
\label{def:gate-group}
\lean{MAG.Support.Circuits.KrohnRhodes.GateGroup}
\leanok
\uses{def:gate-type}

The group component of each gate's syntactic monoid.
\end{definition}

\begin{theorem}[Gate groups are solvable]
\label{thm:gate-solvable}
\lean{MAG.Support.Circuits.KrohnRhodes.gate\_group\_solvable}
\leanok
\uses{def:gate-group}

Every basic gate has a solvable group.
\end{theorem}

\begin{theorem}[Circuit Composition Principle]
\label{thm:circuit-composition-principle}
\lean{MAG.Support.Circuits.KrohnRhodes.Circuit\_Composition\_Principle}
\leanok
\uses{thm:gate-solvable}

Standard Boolean circuits have solvable operation groups:
\[
  \forall C,\ \mathrm{IsSolvable}(\mathrm{CircuitOpGroup}(C))
\]
\end{theorem}

%%%%%%%%%%%%%%%%%%%%%%%%%%%%%%%%%%%%%%%%%%%%%%%%%%%%%%%%%%%%%%%%%%%%%%%%%%%%%
\chapter{Syntactic Theory}
\label{chap:syntactic}

\paperref{Appendix A}

This chapter proves the fundamental theorem of syntactic monoids.

\begin{definition}[SyntacticMonoid]
\label{def:syntactic-monoid}
\lean{MAG.Support.Syntactic.Theory.SyntacticMonoid}
\leanok

The quotient $\Sigma^* / {\sim_L}$.
\end{definition}

\begin{definition}[WordProblem]
\label{def:word-problem}
\lean{MAG.Support.Syntactic.Theory.WordProblem}
\leanok

The word problem of a group $G$: $w \in L \Leftrightarrow w.\mathrm{prod} = 1$.
\end{definition}

\begin{theorem}[Word problem syntactic monoid isomorphism]
\label{thm:word-problem-iso}
\lean{MAG.Support.Syntactic.Theory.wordProblem\_syntacticMonoid\_iso\_self}
\leanok
\uses{def:syntactic-monoid, def:word-problem}

The syntactic monoid of the word problem of $G$ is isomorphic to $G$:
\[
  \mathrm{SyntacticMonoid}(\mathrm{WordProblem}_G) \simeq G
\]
\end{theorem}

%%%%%%%%%%%%%%%%%%%%%%%%%%%%%%%%%%%%%%%%%%%%%%%%%%%%%%%%%%%%%%%%%%%%%%%%%%%%%
\chapter*{Appendix: Lean Artifact Map}
\addcontentsline{toc}{chapter}{Appendix: Lean Artifact Map}

\section*{Core Theorems}

\begin{center}
\begin{tabular}{|l|l|l|}
\hline
\textbf{Theorem/Definition} & \textbf{File} & \textbf{Status} \\
\hline
MAG\_P\_neq\_NP (Thm 3.1) & Core/MAG\_P\_neq\_NP.lean & $\checkmark$ sorry-free \\
A5\_Universal\_Barrier (Thm 4.6) & Core/A5\_Barrier.lean & $\checkmark$ sorry-free \\
solvable\_vanishing\_depth (Thm 4.7) & Core/BarringtonBarrier.lean & $\checkmark$ sorry-free \\
Bridge\_P\_neq\_NP (Thm 5.4) & Core/Bridge/Theorem.lean & $\checkmark$ sorry-free \\
BarringtonToy\_P\_neq\_NP (Thm 5.7) & Core/Bridge/ToyInstantiation.lean & $\checkmark$ sorry-free \\
\hline
\end{tabular}
\end{center}

\section*{Support Modules}

\begin{center}
\begin{tabular}{|l|l|l|}
\hline
\textbf{Module} & \textbf{File} & \textbf{Status} \\
\hline
Barrington algebraic content & Support/Barrington/Complete.lean & $\checkmark$ sorry-free \\
Barrington arithmetic & Support/Barrington/Arithmetic.lean & $\checkmark$ sorry-free \\
Krohn-Rhodes solvability & Support/Circuits/KrohnRhodes.lean & $\checkmark$ sorry-free \\
Syntactic monoid theory & Support/Syntactic/Theory.lean & $\checkmark$ sorry-free \\
\hline
\end{tabular}
\end{center}

\section*{Verification Command}

Entry point: \texttt{Main.lean}

To verify: \texttt{lake build}

%%%%%%%%%%%%%%%%%%%%%%%%%%%%%%%%%%%%%%%%%%%%%%%%%%%%%%%%%%%%%%%%%%%%%%%%%%%%%
% Bibliography
%%%%%%%%%%%%%%%%%%%%%%%%%%%%%%%%%%%%%%%%%%%%%%%%%%%%%%%%%%%%%%%%%%%%%%%%%%%%%

\bibliographystyle{alpha}
\bibliography{references}


\end{document}
